\documentclass[12pt]{article}
\usepackage{bm}       % bold mathematic symbols

% symbol used for sqrt(-1)
\newcommand{\Ci}{{\rm i}}

% general Lorentz boost and rotation
\newcommand\boost[1]{\ensuremath{ {\bf H}^{#1}_{\bm{\hat m}}(\beta) }}
\newcommand\rotat[1]{\ensuremath{ {\bf U}^{#1}_{\bm{\hat n}}(\phi) }}

\newcommand\pauli[1]{\ensuremath{ \bm{\sigma}_{\rm #1} }}

\begin{document}

\section{Introduction}

Quaternions may be considered an generalization of complex numbers,
and are best known by the fundamental formula of quaternion algebra,
\begin{equation}
i^2=j^2=k^2=ijk=-1
\end{equation}
which is said to have been carved into the stone of the Brougham
bridge by the famous mathematician, William Rowan Hamilton, when the
idea occured to him.

An arbitrary quaternion, $\bm{U}$, is given by the linear combination,
$\bm{U}=u_0,+u_1i+u_2j+u_3k$, $u_i\in\Re$.  Defined in this way, the
multiplicative quaternion group may be said to be isomorphic with the
$2\times2$ unitary tranformations, SU(2).

It is also possible to define the quaternion such that it is isomorphic
with $2\times2$ Hermitian matrices.  In this case, $i^2=j^2=k^2=ijk=1$,
and an arbitrary quaternion, $\bm{H}$, may be represented by the
linear combination:
\begin{equation}
\bm{H}=h_0\pauli{0} + h_1\pauli{1}
	+ h_2\pauli{2} + h_3\pauli{3}, \hspace{5mm} h_i\in\Re
\end{equation}
In this notation, $\pauli{0}$ may be understood as the identity matrix,
$\bm{I}$, and $\pauli{1-3}$ as the Pauli spin matrices:
\begin{eqnarray}
\pauli{1} = \left( \begin{array}{cc}
1 & 0 \\
0 & -1 
\end{array}\right)
&
\pauli{2} = \left( \begin{array}{cc}
0 & 1 \\
1 & 0 
\end{array}\right)
& 
\pauli{3} = \left( \begin{array}{cc}
0 & -\Ci \\
\Ci & 0
\end{array}\right).
\label{pauli}
\end{eqnarray}
From the above definitions, it can easily be shown that the Pauli spin
matrices have the following properties:
\begin{eqnarray}
\pauli{i}^2 = \bm{I}, & \pauli{i}\pauli{j} = -\pauli{j}\pauli{i} = \Ci\pauli{k}
\end{eqnarray}
where $\{i,j,k\}$ is chosen from cyclic permutations of $\{1,2,3\}$.

Quaternions may be represented as a four-vector, $\bm{H}=[h_0,h_1,h_2,h_3]$.
A very useful notation represents the quaternion as a scalar plus a 
three vector, 
\begin{equation}
\bm{H}=[h+\bm{h}] = h\sigma{0} + \bm{h\cdot\sigma}.
\end{equation}
Here, $\bm{h}=(h_1,h_2,h_3)$ and $\bm\sigma$ is a 3-vector whose
components are the Pauli spin matrices as defined above.

Quaternions with complex coefficients are known as a biquaternions and
will not be further discussed in this paper.

\section{Some Algebraic Derivations}

\subsection{Quaternion Multiplication}
As with matrices, quaternion multiplication is non-commutative.  It is
useful to derive the result of the multiplication of two quaternions in some
detail. Starting with $\bm{A}=[a+\bm{a}]$ and $\bm{B}=[b+\bm{b}]$
\[
\bm{A}\bm{B}=
(a_0\pauli{0} + a_1\pauli{1} + a_2\pauli{2} + a_3\pauli{3})
(b_0\pauli{0} + b_1\pauli{1} + b_2\pauli{2} + b_3\pauli{3}) \\
= a_0b_0\pauli{0}\pauli{0} + a_0b_1\pauli{0}\pauli{1}
	+ a_0b_2\pauli{0}\pauli{2} + a_0b_2\pauli{0}\pauli{2} \\
+ a_1b_0\pauli{1}\pauli{0} + a_1b_1\pauli{1}\pauli{1}
	+ a_1b_2\pauli{1}\pauli{2} + a_1b_2\pauli{1}\pauli{2}
\]

\section{Isomorphism with Jones matrices}

An arbitrary $2\times2$ complex, or Jones, matrix may be represented
by its polar decomposition,
\begin{equation}
{\bf J} = J \; \boost{} \rotat{},
\label{eqn:polar_decomposition}
\end{equation}
where $J=(\det{\bf J})^{1/2}$,
\begin{eqnarray}
\boost{} &=& \exp(\bm{\sigma\cdot}\bm{\hat m}\beta)
        = [\cosh\beta,\sinh\beta\;\bm{\hat m}],
\label{eqn:boost} \\
\rotat{} &=& \exp(i\bm{\sigma\cdot}\bm{\hat n}\phi)
        = [\cos\phi,i\sin\phi\;\bm{\hat n}],
\label{eqn:rotation}
\end{eqnarray}
and $\bm{\hat m}$ and $\bm{\hat n}$ are real-valued unit 3-vectors.
The matrices, \rotat{}\ and \boost{}, are unimodular (ie. have unit
determinant) and are known as the axis-angle representation of unitary
and Hermitian transformations, respectively.

A matrix may also be represented by ${\bf J} = J \; \rotat{\prime}
\boost{\prime}$, noting that $\rotat{\prime}\ne\rotat{}$ and
$\boost{\prime}\ne\boost{}$ (owing to the non-commutivity of matrix
multiplication).  However, as shown in Section~\ref{sec:root}, the
decomposition shown in Eq.~\ref{eqn:polar_decomposition} provides greater
advantages.



\end{document}


